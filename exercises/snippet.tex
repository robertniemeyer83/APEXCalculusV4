\documentclass{amsart}
\usepackage{mathtools}
\begin{document}
	
%lagrange multipliers section
{
$f(x,y) = x+y$, $3x^2+y^2 = 3$
}
{
solution
}

{
$f(x,y) = 3x^2 +y^2$, $x+y = 1$.  
}
{
solution
}

{
$f(x,y) = \sin x+ \cos y - x$, $x^2+y^2 = 1$
}
{
solution
}

{
$f(x,y) = \sin x+ \cos y  - x$, $x+y = 2$.  
}
{
solution
}

{
$f(x,y) = xe^y$, $x^2 - 2y^2 = 4$
}
{
solution
}

{
$f(x,y,z) = xy +yz$, $x+y = 2$, $y+z = -1$
}
{
solution
}

{
Suppose a company makes a profit $p(x,y) = 7x^2+ 5$ and the production cost is $c(x,y) = 3x+2y$.  Suppose we want to keep $c(x,y)$ at $\$500, 000.00$.  Determine the maximum profit subject to the stated constraint.
}
{
solution
}

%matrix concept problems
{
Can we compute the determinant of 
\[A = 
\begin{bmatrix}
	3 & 2 & 1\\
	1 & 2 & 3
\end{bmatrix}?
\]
}
{
solution
}

{
Can we add the following two matrices? Can we multiply the two matrices?
\[A = 
\begin{bmatrix}
	3 & 2 & 1\\
	1 & 2 & 3
\end{bmatrix}
\quad 
B = 
\begin{bmatrix}
 1 & 2\\
 3 & 4\\
 5 & 9
\end{bmatrix}?
\]
}
{
solution
}

{
What are the dimensions of the following products of matrices?

$AB$ and $BA$ 
\[ A = 
\begin{bmatrix}
	3 & 2 & 1\\
	1 & 2 & 3
\end{bmatrix}
\quad 
B = 
\begin{bmatrix}
 1 & 2\\
 3 & 4\\
 5 & 9
\end{bmatrix},
\]

\[A = 
\begin{bmatrix}
	-3 & 0 & -1\\
	0 & 7 & -9\\
	4 & 5 & 0
\end{bmatrix}
\quad 
B = 
\begin{bmatrix}
 12 & 2\\
 0 & -4\\
 5 & -1
\end{bmatrix}?
\]


}
{
solution
}

{
What can you state about multiplication of matrices? Use technical terms and be  in your answers.

%determinant computations
{
Compute $\det A$ where $A$ is as follows
\[A=
\begin{bmatrix}
	0 & 1 & 0 \\
	1 & 0 & 1 \\
	0 & 1 & 0
\end{bmatrix}
\]
}
{
solution
}

{
Compute $\det A$ where $A$ is as follows
\[A=
\begin{bmatrix}
	-1 & 1 & 5 \\
	1 & 3 & 2 \\
	0 & 4 & 6
\end{bmatrix}
\]
}
{
solution
}

{
Compute $\det A$ where $A$ is as follows
\[A=
\begin{bmatrix}
	1 & 2 & 3 \\
	4 & 5 & 6 \\
	7 & 8 & 9
\end{bmatrix}
\]
}
{
solution
}

{
Compute $\det A$ where $A$ is as follows
\[A=
\begin{bmatrix}
	0 & 2 & 5 \\
	7 & 8 & 9 \\
	0 & 0 & -1
\end{bmatrix}
\]
}
{
solution
}

{
Compute $\det A$ where $A$ is as follows
\[A=
\begin{bmatrix}
	1 & 2 & 5 & -3 \\
	9 & -2 & 0 & 1 \\
	-7 & 0 & -3 & 4
\end{bmatrix}
\]
}
{
solution
}


%more matrix concepts

{
If $A$ is an $n\times n$ matrix, what would you expect the determinant of 
\[A =
\begin{bmatrix}
	a_{11} & 	a_{11} & \dots & a_{1, n} \\
0	& a_{2,2} & \dots & a_{2,n} \\
\vdots & \vdots  &\ddots & \vdots \\
0 & 0 & \dots &a_{n,n}
\end{bmatrix}
	\]
to be?
}
{
solution
}

%Hessian problems
{
Compute the Hessian of the following function.
\[f(x,y,z) = xyz\]
}
{
Solution
}

{
Compute the Hessian of the following function.
\[f(x,y,z) = \sin x + \sin y +\sin z\]
}
{
Solution
}

{
Compute the Hessian of the following function.
\[f(x,y) = \sin(xy)\]
}
{
Solution
}

{
Compute the Hessian of the following function.
\[e^y\ln x\]
}
{
}

{
Determine the critical points of the following function.
\[f(x,y,z) = xyz\]
}
{
}

{
Determine the critical points of the following function.
\[f(x,y,z) = \sin x + \sin y +\sin z\]
}
{
}

{
Determine the critical points of the following function.
\[f(x,y) = \sin(xy)\]
}
{
}

{
Determine the critical points of the following function.
\[f(x,y) = e^y\ln x\]
}
{
}

{
Determine whether or not the Hessian $hess_f(a)$ for each of the previous functions is positive definite, negative definite or neither.
}
{

}
}

{
Determine the critical points of each of the previous functions.
}
{
}

{
Using the general 2nd derivative test and the functions $f(\vec x)$ from the previous problem, determine whether or not $f(\vec a)$ is a local minimum or maximum for $f(\vec x)$ or if the test is inconclusive for the critical point $\vec a$.
}
{
}

%Hessian concept
{
Define positive definite and negative definite.
}
{
}

{
Explain why $x^2 - y^2s$ is not positive definite.
}
{
}

{
Explain how 

\begin{align}
	A(a)&=\begin{bmatrix}
				\frac{\partial^2 f(\vec a)}{\partial x_1^2} & \frac{\partial^2 f(\vec a)}{\partial x_1\partial x_2} & \ldots & \frac{\partial^2 f(\vec a)}{\partial x_1 \partial x_n} \\
				\frac{\partial^2 f(\vec a)}{\partial x_2 \partial x_1} & \frac{\partial^2 f(\vec a)}{\partial x_2^2}&  \ldots & \frac{\partial^2 f(\vec a)}{\partial x_2 \partial x_n} \\
				\vdots & \vdots & \ddots & \vdots \\
				\frac{\partial^2 f(\vec a)}{\partial x_m \partial x_1} & \frac{\partial^2 f(\vec a)}{\partial x_m\partial x_2} &\ldots & \frac{\partial^2 f(\vec a)}{\partial x_n^2}			
			\end{bmatrix}
\end{align}

may be positive definite but $\vec a $ not a critical point of $f(\vec x)$
}
{
}

%Change of variable problems
{
Which of the following transformations are one-to-one?

\begin{itemize}
	\item $T(u,v) = (u^2-v^2, u+v)$
	\item $T(u,v) = (u^2 - v^2, u+v)$, $u\geq 0$, $v\leq 0$.
	\item  $T(u,v) = (\sin (u+v), \cos (u-v))$
	\item $T(u,v) = (\sin (u+v), \cos (u-v))$, $\frac{\pi}{6}\leq u\leq \frac{\pi}{3}$ and $\frac{\pi}{6} \leq v\leq  \frac{\pi}{3}$.
\end{itemize}
}
{
}

{
Determine whether or not the Jacobian vanishes at any points for the above transformations.
}
{
}

%concepts for jacabians
{
Does a change of variable given by a transformation $T(u,v) = (x(u,v), y(u,v))$ need to be a linear transformation?
}
{
}

%calc problems for jacobians
Compute Jacobian of the following transformation

{
$T(u,v) = \left(\frac{u+1}{v-1}, u^2-v^2\right)$
}
{
}

{
$T(u,v,w) = \left (w-u + v, uv, ve^u\right)$, $u,v,w\geq 0$.
}
{
}

Compute the following 
{
$\iint_A \sqrt{x+y} \, dxdy$ $A=[0,1]\times [0,1]$.
}
{
}

If $X= [0,1]\times [0,1]$, determine the region one gets under the following transformations.
{
$T(u,v) = (u-v,u+v)$
}
{
}

{
$T(u,v) = (u-2v,u+v)$
}
{
}

{
$T(u,v) = (2u-v,u+v)$
}
{
}

{
$T(u,v) = (u-v,2u+v)$
}
{
}

{
$T(u,v) = (u-v,u+2v)$
}
{
}

{
What can you say about the effect of $T(u,v) = (au-bv,cu+dv)$?
}
{
}

{
Define a rotation that will rotate a region in the plane by $\pi/6$ radians.
}
{
}

{
How does $(u-v,u+v)$ compare to 
\[
\begin{bmatrix*}[r]
	1 & -1\\
	1 & 1
\end{bmatrix*}
\begin{pmatrix}
	u \\ 
	v
\end{pmatrix}
\]
}
{
}


{
How does 

\[
\begin{bmatrix*}[r]
	1 & -1\\
	1 & 1
\end{bmatrix*}
\]

compare to 


\[
\begin{bmatrix*}[r]
	\cos \frac{\pi}{4}  & -\sin\frac{\pi}{4}\\
	\sin \frac{\pi}{4} & \cos \frac{\pi}{4}
\end{bmatrix*}?
\]
}
{
}

{
What is the effect of 

\[
T_1 =
\begin{bmatrix*}[c]
1 & 0 &0 \\
0 & \cos \theta & -\sin \theta \\ 
0 & \sin \theta & \cos \theta 
\end{bmatrix*}
\]

\[
T_2 =
\begin{bmatrix*}[c]
\cos \theta & 0 & -\sin \theta \\ 
0 & 1 & 0 \\ 
\sin \theta &0  & \cos \theta 
\end{bmatrix*}
\]

and

\[
T_3 =
\begin{bmatrix*}[c]
 \cos \theta & -\sin \theta & 0 \\ 
\sin \theta & \cos \theta & 0 \\
0 & 0 & 1
\end{bmatrix*}
\]
on vectors $(u,v,w)\in\mathbb{R}^3$?

}
{
}



\end{document}
