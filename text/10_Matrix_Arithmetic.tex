\section{Matrix Arithmetic}\label{sec:matrix_arithmetic}

This version of \apex Calculus III V4 does not assume any background in linear algebra.  As such, we include this section to give the reader the necessary background for understanding future computations in Section \ref{sec:deriv_matrix} in a broader context.  

\bigskip

In this section, we give the basics of matrix arithmetic.  A matrix is a two-dimensional array of objects (typically numbers) and is written as follows:
\[
A_{m \times n} = 
\begin{bmatrix}
a_{1,1} & a_{1,2} & \dots & a_{1,n}\\
a_{2,1} & a_{2,2} &\dots & a_{2,n}\\
\vdots & \vdots &\ddots&  \vdots \\
a_{m,1} & a_{m,2}  & \dots  & a_{m,n}
\end{bmatrix}
\]

The matrix $A_{m\times n}$ is a two-dimensional array of values where each entry $a_{i,j}$ is indexed by two positive integers $i$ and $j$, where $1\leq i\leq m$ and $1\leq j \leq n$.  There are $m$ many rows and $n$ many columns in the matrix.  What we want to understand is under what conditions can we add and subtract two matrices.  For addition and subtraction of matrices to make sense, both matrices must have the same dimensions, namely $m\times n$.  

%%put in example of adding matrices.

%%then show how to multiply a matrix by a vector

%% relate this back to the dot product

%%Then show how to multiply matrices

%%introduce n\times n identity matrix and invertible matrices.

%%show how to compute the determinant of an $n\times n$ matrix

%%show that an $n\times n$ matrix is invertible if and only if det A neq 0

%% 