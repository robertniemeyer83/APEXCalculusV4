\section{Matrix Arithmetic}\label{sec:matrix_arithmetic}

This version of \apex Calculus III V4 does not assume any background in linear algebra.  As such, we include this section to give the reader the necessary background for understanding future computations in Section \ref{sec:deriv_matrix} in a broader context.  

\bigskip

In this section, we give the basics of matrix arithmetic.  A matrix is a two-dimensional array of objects (typically numbers) and is written as follows:
\[
A_{m \times n} = 
\begin{bmatrix}
a_{1,1} & a_{1,2} & \dots & a_{1,n}\\
a_{2,1} & a_{2,2} &\dots & a_{2,n}\\
\vdots & \vdots &\ddots&  \vdots \\
a_{m,1} & a_{m,2}  & \dots  & a_{m,n}
\end{bmatrix}
\]

\mnote{.65}{\textbf{Note:} We have emphasized the dimensions of the matrix by including the subscript $m\times n$.  When the dimension of the matrix is either known implicitly or explicitly illustrated by describing the matrix by listing all of its entries, then we will not indicate the dimensions as a subscript.}

The matrix $A_{m\times n}$ is a two-dimensional array of values where each entry $a_{i,j}$ is indexed by two positive integers $i$ and $j$, where $1\leq i\leq m$ and $1\leq j \leq n$.  There are $m$ many rows and $n$ many columns in the matrix.  

One way to view a matrix $A_{m\times n}$ is in terms of its $m$-many row vectors, or viewing each row of a matrix as a vector in its own right.  Similarly, one can view a matrix as comprised of $n$-many column vectors, each column vector having $m$-many entries. 
\bigskip

What we want to first understand is under what conditions can we add and subtract two matrices.  For addition and subtraction of matrices to make sense, both matrices must have the same dimensions, namely $m\times n$. \\ 

\example{ex_adding matrices}{Adding and Subtracting Matrices}
{Consider the matrices $A$ and $B$

\begin{align}  
A= 
\begin{bmatrix}
	3 & \pi & e \\
	-1 &0 & \frac{3}{2}\\
	0 & 2.2 & -2
\end{bmatrix}
&
\quad B=
\begin{bmatrix}
	3 & 0 & 2 \\
	-3.1 & 1 & \frac{5}{2}\\
	0 & -1.7 & -1	
\end{bmatrix}.
\end{align}

Compute $A+B$ and $A-B$.
}
{
We add elements of a matrix by their position in the matrix.  By that, we mean that $a_{i,j}$ and $b_{i,j}$ are added for each $i,j$:
\begin{align}  
A+B &=  
\begin{bmatrix}
	3 & \pi & e \\
	-1 &0 & \frac{3}{2}\\
	0 & 2.2 & -2
\end{bmatrix}
+ 
\begin{bmatrix}
	3 & 0 & 2 \\
	-3.1 & 1 & \frac{5}{2}\\
	0 & -1.7 & -1	
\end{bmatrix}\\
&= \begin{bmatrix}
	6 & \pi &e + 2 \\
	-4.1 & 1 & 4 \\
	0 & 0.5 & -3
\end{bmatrix}.
\end{align}

Subtraction of matrices is similarly performed.  
\begin{align}  
A+B &=  
\begin{bmatrix}
	3 & \pi & e \\
	-1 &0 & \frac{3}{2}\\
	0 & 2.2 & -2
\end{bmatrix}
-
\begin{bmatrix}
	3 & 0 & 2 \\
	-3.1 & 1 & \frac{5}{2}\\
	0 & -1.7 & -1	
\end{bmatrix}\\
&= \begin{bmatrix}
	0 & \pi &e - 2 \\
	2.1 & -1 & -1 \\
	0 & 3.9 & -1
\end{bmatrix}.
\end{align}
}\\


Multiplying matrices is a bit more complicated.  If we have matrices $A_{m\times n}$ and $B_{r\times s}$, then the product of these two matrices will only be defined if $r=n$.  In general, if $r=n$, then $A_{m\times n}B_{r\times s} = C_{m\times s}$.  In other words, the product of two matrices with indices matching up appropriately will be a matrix with the aforementioned dimensions, $m\times s$.  We now describe how to compute the product of two matrices.

Let $A_{m\times n}$ and $B_{n\times s}$ be two matrices.   Then $A_{m\times n}B_{n\times s} = C_{m\times s}$ where 

\[c_{i,j} = \sum_{k=1}^n a_{i,k}b_{k,j}.\]

Since the number of entries in a single row of the matrix $A$ is exactly the number of entries in a single column of matrix $B$, we see that $c_{i,j}$ is simply the dot product of the row vector $\vec{a}_{i}$ with the column vector $\vec{b}_j$.  We now provide an example of the product of two matrices.\\

\example{ex_matrix_multiplication}{The Product of Two Matrices}
{
Consider the following matrices.

\begin{align}
	A = \begin{bmatrix}
		2, -1, 0\\
		\pi, 3, e
	\end{bmatrix}
	&\quad B = \begin{bmatrix}
		1 & 9 \\
		-1 & 2\\
		7 & 0		
	\end{bmatrix}
\end{align}
Compute the product $AB$.
}
{
We know that the product of these two matrices will be a matrix with 2 rows and 2 columns.  The entry $c_{ij}$ of the matrix $C_{2\times 2}$ will be, as described above:

\[c_{i,j} = \sum_{k=1}^3 a_{i,k}b_{k,j}.\]

For $i=1$ and $j=2$, for example:

\begin{align}
c_{1,2} &= \sum_{k=1}^3 a_{1,k}b_{k, 2}\\	
			 &= 2\cdot 0+ -1\cdot 2+0\cdot 0 \\
			 &= -2.
\end{align}

Continuing, we see that

\[C=
\begin{bmatrix}
	3 & -2\\
	\pi + 3 +e &9\pi+6
\end{bmatrix}.
\]
}\\

You may have noticed that one can reframe matrix multiplication as the dot product of a row vector and a column vector.  The dot product of a row vector of $A$ and a column vector of $B$ yields a single entry in $C$.  In terms of our notation, this is 

\[ c_{i,j} = \vec{a}_i \cdot\vec{b}_j.\]

Given our method for computing the product of two matrices, we can easily determine the following product of an $m\times n$ matrix $A$ and an $n\times 1$ column vector $\vec b$.

\[
	A_{m\times n}\vec{b} =
	\begin{bmatrix}
		a_{1,1} & a_{1,2} & \dots & a_{1,n}\\
		a_{2,1} & a_{2,2} &\dots & a_{2,n}\\
		\vdots & \vdots &\ddots&  \vdots \\
		a_{m,1} & a_{m,2}  & \dots  & a_{m,n}
	\end{bmatrix}
	\vec{b}
\]

We expect such a product to yield yet another vector $C_{m\times 1}$, which, as the notation indicates, is simply a matrix with dimensions $m\times 1$.  Put more explicitly, 

\[C_{m\times 1} = 
\begin{bmatrix}
	\vec{c}_1 \cdot \vec b\\
	\vec{c}_2 \cdot \vec b\\
	\vdots \\
	\vec{c}_m \cdot \vec b
\end{bmatrix}.
\]

We close this section by  introducing the determinant of a square matrix $A_{m\times m}$.  This is simply a number that we can calculate from  a matrix.  It has significance beyond Calculus III, but we will not elaborate on such in this text.  

We have already given a technique for calculating the determinant of a $3\times 3$ matrix in Section \ref{sec:cross_product}, but such a technique does not generalize to larger square matrices.  

We begin by giving the reader a formula for calculating the determinant of a matrix $A_{2\times 2}$.

\definition{def_determinant_2x2_matrix}{The Determinant of a $2\times 2$ matrix}
{Let $A$ be a $2\times 2$ matrix.  We define the determinant of $A$ as follows.

\[\det A = a_{1,1}a_{22} - a_{21}a_{12}.\]
}

We then define the determinant of an $m\times m$ matrix as follows. 

\definition{def_determinant_nxn}{Determinant of an $m\times m$ matrix}
{Let $A$ be an $m\times m$ matrix.  Then, 
\[\det A = \sum_{j=1}^m (-1)^{1+j} a_{1j}\det M^{1,j},\]

\noindent where $M^{1,j}$ is an $(m-1)\times (m-1)$ square matrix constructed from $A$ by eliminating the first row and $j$th column of $A$.
}

\mnote{.5}{\textbf{Note:} We will see below that this is not the most general definition of determinant.  In actuality, one can make the definition even more general, but such a discussion is better suited for a course on abstract algebra.  The interested reader is encouraged to ask their instructor for further reading.}

You will notice in the preceding definition that the determinant of a square matrix is recursively defined, meaning that in order to compute the determinant of an $m\times m$ matrix, one must compute the determinant of an $(m-1)\times (m-1)$ matrix.\\

\example{ex_determinant_3x3_matrix}{The Determinant of a $3\times 3$ matrix}
{Compute the determinant of the following matrix.

\[A = 
\begin{bmatrix}
3 &-5 & 2\\
0 & 1 & 4\\
8 & -5 & 0
\end{bmatrix}
\]
}
{
\begin{align*}
	\det A &= \sum_{j=1}^3 (-1)^{1+j}a_{1,j} \det M^{1,j} \\
				&= a_{1,1}
				\det \begin{bmatrix}
					1 & 4\\
					-5 & 0
				\end{bmatrix}
				- a_{1,2}
				\det \begin{bmatrix}
					0 & 4\\
					8 & 0 
					\end{bmatrix}
				+ a_{1,3}
				\det \begin{bmatrix}
					0 & 1\\
					8 & -5
				\end{bmatrix}\\
				&= 3(1\cdot 0 + 5\cdot 4)- (-5)(0\cdot 0 - 8\cdot 4)+ 2(0\cdot (-5)-8\cdot 1)\\
				& = 60+5(-32)+2(-8)\\
				& = -116.
\end{align*}
}

What is presented in Definition \ref{def_determinant_nxn} is the definition of a determinant as computed by \sword{expanding along the first row.}  In reality, one is allowed to \sword{expand along any row or column} of the matrix $A$.  We give the formal general definition as follows.

\definition{def_general_determinant_nxn}{Determinant of an $m\times m$ matrix: a general formula}
{Let $A$ be an $m\times m$ matrix.  For any fixed $i$, $1\leq i\leq m$,  we may define the determinant of $A$ to be,
 
\[\det A = \sum_{j=1}^m (-1)^{i+j} a_{ij}\det M^{i,j},\]

\noindent where $M^{1,j}$ is an $(m-1)\times (m-1)$ square matrix constructed from $A$ by eliminating the first row and $j$th column of $A$.

Similarly, for any fixed $j$, $1\leq j\leq n$, we may define the determinant of $A$ to be,

\[\det A = \sum_{i=1}^m (-1)^{i+j} a_{ij}\det M^{i,j},\]
}

\example{ex_general_determinant_3x3}{The determinant of a $3\times 3$ matrix}
{Compute the determinant of the matrix $A$ in Example \ref{ex_determinant_3x3_matrix} by expanding along the first column.}
{This amounts to letting $j=1$ and letting $i$ vary from $1$ to $3$ in the general formula given in Definition \ref{def_general_determinant_nxn}.}


%%then show how to multiply a matrix by a vector

%% relate this back to the dot product

%%Then show how to multiply matrices

%%introduce n\times n identity matrix and invertible matrices.

%%show how to compute the determinant of an $n\times n$ matrix

%%show that an $n\times n$ matrix is invertible if and only if det A neq 0

%% 